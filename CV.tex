\documentclass[margin,line]{CV}

\usepackage{enumitem}
\usepackage[colorlinks = true,
linkcolor = blue,
urlcolor  = blue,
citecolor = blue,
anchorcolor = blue]{hyperref}
\oddsidemargin -.5in
\evensidemargin -.5in
\textwidth=6.0in
\itemsep=0in
\parsep=0in
% if using pdflatex:
%\setlength{\pdfpagewidth}{\paperwidth}
%\setlength{\pdfpageheight}{\paperheight} 

\newenvironment{list1}{
  \begin{list}{\ding{113}}{%
      \setlength{\itemsep}{0in}
      \setlength{\parsep}{0in} \setlength{\parskip}{0in}
      \setlength{\topsep}{0in} \setlength{\partopsep}{0in} 
      \setlength{\leftmargin}{0.17in}}}{\end{list}}
\newenvironment{list2}{
  \begin{list}{$\bullet$}{%
      \setlength{\itemsep}{0in}
      \setlength{\parsep}{0in} \setlength{\parskip}{0in}
      \setlength{\topsep}{0in} \setlength{\partopsep}{0in} 
      \setlength{\leftmargin}{0.2in}}}{\end{list}}


\begin{document}

\name{Eric Nunes \vspace*{.1in}}

\begin{resume}
\section{\sc Contact Information}
\vspace{.05in}
\begin{tabular}{@{}p{2in}p{4in}}
699 S. Mill Avenue            & {\it Voice:}  (315) 439-3089\\                    
Arizona State University & {\it E-mail:}  \href{mailto:enunes1@asu.edu}{\bf enunes1@asu.edu}\\       
Tempe, AZ  85281 USA  & {\it website:} \href{https://efnunes.github.io/}{\bf efnunes.github.io} \\     
\end{tabular}


%\section{\sc Research Interests}
%Artificial Intelligence, Machine learning and statistical methods for large datasets, data-driven cyber security, security informatics.

\section{\sc Education}
\href{http://www.asu.edu/}{\bf Arizona State University}, Tempe, Arizona USA\\
%{\em Department of Statistics} 
\vspace*{-.1in}
\begin{list1}
\item[] Ph.D. Computer Engineering (GPA: 4.0/4.0), May 2018
\begin{list2}
\vspace*{.05in}
\item Dissertation Topic:  ``Reasoning about Cyber Threat Actors'' 
%\item Dissertation Topic:  ``Hierarchical Models for Multiple Ratings
%  in Performance-Based\\ \hspace*{1.23in} Student Assessments.'' 
\item Advisor:  Paulo Shakarian
\end{list2}
\end{list1}

\href{http://www.syr.edu/}{\bf Syracuse University}, Syracuse, New York USA\\
%{\em Department of Mathematics and Statistics} 
\vspace*{-.1in}
\begin{list1}
\item[] M.S. Electrical Engineering,  May 2012
\end{list1}

\href{http://archive.mu.ac.in/}{\bf University of Mumbai}, Mumbai, India\\
%{\em Department of Mathematics and Statistics} 
\vspace*{-.1in}
\begin{list1}
\item[] B.S. Electronics and Telecommunication,  June 2010
\end{list1}


\section{\sc Honors and Awards} 
\begin{itemize}[leftmargin =*]
	
	
\item IEEE International Conference on Data Intelligence and Security (ICDIS), 2018 \textbf{Best Poster Award} for ``DARKMENTION: Reasoning about enterprise-related external cyber threats using a rule-learning approach".

\item IEEE/ACM International Symposium on Foundations of Open Source Intelligence and Security Informatics (FOSINT-SI), 2016 \textbf{Best Paper Award} for ``Argumentation Models for Cyber Attribution". 

\item ``Systems and Methods for Data Driven Malware Task Identification"-- Selected for TechConnect
2016 Innovation Showcase.

\item Business Category - Most commercial potential winner (Idea: Weight Estimation from Anthropometric features), Medical Center of The Americas Foundation, 2014 (\$1000).

\item Graduate Scholarship to pursue M.S. at Syracuse University (2010 - 2012)
\end{itemize}
\section{\sc Academic Experience}
{\bf Arizona State University}, Tempe, Arizona USA\\
{\bf Graduate Research Assistant \href{http://cysis.engineering.asu.edu/}{(CySIS Lab)}} \hfill {\bf August 2014 - May 2018}\\
\textit{\underline{Tools:} Python, Spark, PostgreSQL, Prolog, tcpflow.}\\
\begin{enumerate}
	\item {\bf{Proactive Cyber-threat Intelligence}} 
	\begin{itemize}
		\item Developed an operational system for cyber threat intelligence gathering from darknet and deepnet sites. 
		
		\item The system employs data mining and machine learning techniques to collect information from hacker forum discussions and marketplaces offering products and services focusing on malicious hacking. 
		
		\item Currently, this system collects high-quality cyber threat warnings each week. These threat warnings include information on newly developed malware and exploits. 
		
		\item Developed data analysis tools to identify At-risk systems, vulnerabilities likely to be exploited by threat actors, understand the social dynamics in hacker communities. 
		
		\item {\em Relevant publications:} [B-1, J-1, C-10, C-9, C-7]
	\end{itemize}
	
	
	\item {\bf{Reasoning framework for Cyber-attribution}} 
	\begin{itemize}
		\item Proposed a knowledge representation - machine learning (KR-ML) framework to reason about threat actors. 
		
		\item The framework combines an argumentation model based on DeLP (Defeasible Logic Programming) and machine learning classifiers to evaluate evidence and reason about actors responsible for an attack.
		
		\item The framework was evaluated by building a dataset from the capture-the-flag event held at DEFCON -- 10 million network attacks. 
		
		\item Achieved higher accuracy than previously reported approaches (evaluated on the same dataset) that rely on machine learning classifiers alone---a jump from 37\% to 64.5\%.
		
		\item {\em Relevant publications:} [B-2, J-3, C-3, C-5, C-6, C-8, BC-1]
		
	\end{itemize}
	
	
	\item {\bf{Malware task identification}} 
	\begin{itemize}
		\item Developed a novel cognitive learning model to identify tasks (e.g. logging keystrokes, recording video, establishing remote access, etc.) that the malware was designed to perform. 
		\item The proposed model was tested on different malware collections - including mutated and encrypted malware samples. 
		\item The model outperformed standard machine learning approaches in identifying the tasks.
		\item {\em Relevant publications:} [J-2, C-1, C-2, C-4]
	\end{itemize}
	

\end{enumerate}

{\bf Dartmouth College}, Hanover, New Hampshire USA\\
{\bf Research Associate \href{https://www.dartmouth.edu/~rhg/}{(Brain Engineering Lab)}} \hfill {\bf June 2012 - July 2014}\\
\textit{\underline{Tools:} MATLAB, C++, OpenCV.}
\begin{itemize}
	\item Learning representations for Object recognition and localization from image and video datasets using biologically inspired algorithms.
	\item Proposed a supervised object recognition algorithm that achieves corresponding classification rates in comparison with standard machine learning approaches - at a fraction of the time and space costs.
\end{itemize}
 

{\bf SUNY Upstate Medical University}, Syracuse, New York USA\\
{\bf Research Assistant} \hfill {\bf May 2011 - June 2012}\\
\textit{\underline{Tools:} MATLAB, C++.}\\
Registering Multi-Spectral Retinal images to find features and points of interest to estimate the abundance of Oxygen saturation in the blood vessels in retinal images to diagnose retinal disorders.

\section{\sc Patents}
\begin{itemize}[leftmargin =*]
	
\item ``Systems and Methods for Third Party Risk Assessment." U.S. Provisional Patent: 62/668,871, 2018.

\item ``Systems and Methods for predicting which software vulnerabilities will be exploited by malicious hackers to prioritize for Patching." U.S. Provisional Patent: 62/581,123, 2017.

\item ``Systems and Methods for Data Driven Malware Task Identification." U.S. Patent: 20,160,371,490 (Submitted), 2016. 

\item ``Intelligent darkweb crawling infrastructure for cyber threat intelligence collection."\\U.S. Provisional Patent: 62/409,291, Licensed by CYR3CON, 2016.
\end{itemize}


\section{\sc Publications}
{\bf *B} - {\em Book}, {\bf*J} - {\em Journal}, {\bf *C} - {\em Conference}, {\bf*BC} - {\em Book Chapter}\\

\begin{itemize}[leftmargin =*]

\item {[\bf B-2]}  {\bf E. Nunes}, P. Shakarian, G. Simari, A. Ruef ``Artificial Intelligence Tools for Cyber Attribution", Springer, 2018. 

\item {[\bf B-1]} J. Robertson, A. Diab, E. Marin, {\bf E. Nunes}, V. Paliath, J. Shakarian, P. Shakarian ``Darkweb Cyber Threat Intelligence Mining", Cambridge University Press, 2017. 

\item {[\bf J-1]} J. Robertson, A. Diab, E. Marin, {\bf E. Nunes}, V. Paliath, J. Shakarian, P. Shakarian ``Darknet Mining and Game Theory for Enhanced Cyber Threat Intelligence" The Cyber Defense Review, 2016.

\item {[\bf C-13]} {\bf E. Nunes}, G. Simari, P. Shakarian  ``At-Risk System Identification via Analysis of Discussions on the Darkweb" IEEE APWG Symposium on Electronic Crime Research (eCrime), 2018.

\item {[\bf C-12]} E. Marin, M. Almukaynizi, {\bf E. Nunes}, P. Shakarian ``Community Finding of Malware and Exploit Vendors on Darkweb Marketplaces", IEEE International Conference on Data Intelligence and Security (ICDIS-18), 2018.

\item {[\bf C-11]} M. Almukaynizi, A. Grimm, {\bf E. Nunes}, J. Shakarian, P. Shakarian  ``Predicting Cyber Threats through User Connectivity in Darkweb and Deepweb Forums" ACM Computational Social Science (CSS), 2017.

\item {[\bf C-10]} M. Almukaynizi, A. Grimm, {\bf E. Nunes}, J. Shakarian, P. Shakarian  ``Predicting Cyber Threats through User Connectivity in Darkweb and Deepweb Forums" ACM Computational Social Science (CSS), 2017.

\item {[\bf C-9]} M. Almukaynizi, {\bf E. Nunes}, K. Dharaiya, M. Senguttuvan, J. Shakarian, P. Shakarian  ``Proactive Identification of Exploits in the Wild Through Vulnerability Mentions Online" International Conference on Cyber Conflict (CyCon-US), 2017.


\item {[\bf C-8]} A. Ruef, {\bf E. Nunes}, G. Simari, P. Shakarian  ``Measuring Cyber Attribution In Games" IEEE APWG Symposium on Electronic Crime Research (eCrime), 2017.


\item {[\bf C-7]} {\bf E. Nunes}, A. Diab, A. Gunn, E. Marin, V. Mishra, V. Paliath, J. Robertson, J. Shakarian, A. Thart, P. Shakarian ``Darknet and Deepnet Mining for Proactive Cybersecurity Threat Intelligence" IEEE Conference on Intelligence and Security Informatics (ISI), 2016.

\item {[\bf C-6]} {\bf E. Nunes}, P. Shakarian, G. Simari, A. Ruef  ``Argumentation Models for Cyber Attribution"
IEEE/ACM International Symposium on Foundations of Open Source Intelligence and Security Informatics (FOSINT-SI), 2016 -- {\bf Best Paper Award}.

\item {[\bf C-5]} {\bf E. Nunes}, P. Shakarian, G. Simari ``Toward Argumentation-Based Cyber Attribution"
AAAI Workshop on Artificial Intelligence and Cyber security (AICS), 2016.

\item {[\bf C-4]} {\bf E. Nunes}, C. Buto, P. Shakarian, C. Lebiere, S. Bennati, R. Thomson, H. Jaenisch ``Malware Task Identification: A Data Driven Approach"  IEEE/ACM International Symposium on Foundations of Open Source Intelligence and Security Informatics (FOSINT-SI), 2015.

\item {[\bf C-3]} {\bf E. Nunes},  N. Kulkarni, P. Shakarian, A Ruef, J. Little ``Cyber-Deception and Attribution in Capture-the-Flag Exercises"  IEEE/ACM International Symposium on Foundations of Open Source Intelligence and Security Informatics (FOSINT-SI), 2015.

\item {[\bf C-2]} C. Lebiere, S. Bennati, R. Thomson, P. Shakarian, {\bf E. Nunes}  ``Functional Cognitive Models of Malware Identification" 13th International Conference on Cognitive Modeling (ICCM), 2015.

\item {[\bf C-1]} R. Thomson, C. Lebiere, S. Bennati, P. Shakarian, {\bf E. Nunes}   ``Malware Identification Using Cognitively-Inspired Inference" 24th Conference on Behavior Representation in Modeling and Simulation (BRiMS), 2015.

\item {[\bf BC-1]} {\bf E. Nunes}, N. Kulkarni, P. Shakarian, A Ruef, J. Little ``Cyber-Deception and Attribution in Capture-the-Flag Exercises" (extended version) in Cyber Deception: Building the Scientific Foundation (editors: S. Jajodia, V.S. Subrahmanian, V. Swarup, C. Wang) Springer, 2016.
   
\end{itemize}

\section{\sc Invited talks}
\begin{itemize}[leftmargin =*]
	
\item {\bf Malicious Markets and Forums}\\
Understanding the Dark Web and its Implications for Policy, Virginia Tech’s Executive Briefing Center, Arlington, VA, May 2018.

\item {\bf Cyber-Deception and Attribution in Capture-the-Flag Exercises}\\
The International Information System Security Certification Consortium (ISC2), Phoenix chapter, October, 2016.\\
Army Research Office's Cyber Deception Workshop, Washington, July 2015.

\item {\bf Automatic identification of malware tasks}\\
Cactus-Con, Tempe, Arizona, March, 2015.
\end{itemize}


\section{\sc Professional Experience}

{\bf Member of Technical Staff (MTS 1)}, \href{https://www.paypal.com/us/home}{\bf PayPal}  \hfill {\bf  Starting June 2018}
\begin{itemize}
	
	\item Develop and maintain code to enhance information security capabilities.
	\item Create and maintain infrastructure to support information security capabilities.
	\item Implement machine learning capabilities in new and existing information security products.
	\item Work with information security teams to integrate existing or develop new capabilities.
	\item Create documentation and training for new and existing information security capabilities.
	\item Interpret and solve problems when a user or an automated monitoring system alerts them that a problem exists.
	
\end{itemize}

{\bf Data Scientist}, \href{http://www.cyr3con.com/}{\bf Cyber Reconnaissance Inc. (CYR3CON)}  \hfill {\bf August 2016 - May 2018}\\
Leading a team of developers and analysts to built tools / products for security applications. In particular,
\begin{itemize}
	\item Building a data collection system for Darkweb markets and forums - to collect information regarding discussions and products relating to hacking activities. 
	\item Using the gathered threat intelligence to build learning models for predicting likelihood of exploitation of a vulnerability (vulnerability prioritization).
	\item Providing intelligence on Mobile threats for both Android and iOS applications.
	\item Active Threat Assessment on client systems. 
	\item Named-entity recognition (to determine vulnerable software) using RNN/LSTM seq2seq\\ models.
	\item Developed classification models to classify web scripts as malicious. Visualized the performance of the trained model overtime and analyzed the classification errors for further improvement through Plotly dashboard. Achieved malicious script detection rate of >90%. 
	\item Assist with the expansion of CYR3CON future product features as well as the management
	and development of growing community of users, guiding/assisting them in trials.
\end{itemize}



{\bf Security Automation Intern (Data Science)}, \href{https://www.paypal.com/us/home}{\bf PayPal}  \hfill {\bf May 2017 - August 2017}
\begin{itemize}
    \item Analyzed user login activity using Akamai logs and enriched it with other data feeds such as threat intelligence, merchant data, credential dumps.
	\item Implemented operational Anomaly detection models to detect Account Takeover (ATO) attacks to raise alerts for automated mitigation.  
	\item Visualized ATO attacks in real time on a dashboard in Splunk.
	
\end{itemize}


\section{\sc Technical Skills} 
\begin{itemize}[leftmargin =*]

\item {\bf Machine Learning:} Classification, regression, clustering, anomaly detection, feature engineering, online learning, Experience with deep learning.

\item{\bf Programming Languages:} Python, MATLAB, C++, Prolog, HTML, LaTeX. Familiar with C, R. 

\item{\bf Libraries:} scikit-learn, Weka, Pandas, Elasticsearch, Theano, Caffe. 

\item {\bf Databases:} SQL, PostgreSQL, MongoDB.

\item {\bf Operating System:} Windows, Linux, Mac OS X. 

\item {\bf Tools:} Eclipse, MS Visual Studio, PyCharm. 

\item {\bf Big Data and Cloud:} Splunk, Familiar with Big Data Processing Platforms: Hadoop, Spark and Cloud tools: Amazon S3. 

\end{itemize}


\section{\sc Press}
\begin{itemize}[leftmargin=*]
	
\item 
\href{http://www.defenseone.com/technology/2017/11/which-bugs-will-hackers-exploit-first-machine-learning-promises-better-guess/142621/?oref=d-channeltop}{Which Bugs Will Hackers Exploit First? Machine Learning Promises a Better Guess}, Defense One, November 16, 2017.

\item \href{https://asunow.asu.edu/20160907-solutions-asu-researchers-hacking-hackers-new-approach}{Hacking the hackers}, ASU now: Access, Excellence, Impact. September 7, 2016.

\item \href{https://continuum.cisco.com/2016/08/18/arizona-state-builds-darknet-mining-model-finds-16-zero-days/}{Arizona State Builds Darknet Mining Model, Finds 16 Zero Days}, Cisco Continuum. August 18, 2016.

\item \href{https://www.helpnetsecurity.com/2016/08/10/cyber-threats-underground-markets/}{Over 300 new cyber threats pop up on underground markets each week}, HelpNetSecurity. August 10, 2016.

\item \href{http://www.forbes.com/sites/kevinmurnane/2016/08/08/machine-learning-goes-dark-and-deep-to-find-zero-day-exploits-before-day-zero/#49b6e2706d76}{ Machine Learning Goes Dark And Deep To Find Zero-Day Exploits Before Day Zero}, Forbes. August 8, 2016.

\item \href{https://www.technologyreview.com/s/602115/machine-learning-algorithm-combs-the-darknet-for-zero-day-exploits-and-finds-them/}{ Machine-Learning Algorithm Combs the Darknet for Zero Day Exploits, and Finds Them},
MIT Tech Review. August 5, 2016. ACM TechNews. August 5, 2016.

\end{itemize}

\section{\sc Service}
{\bf Journal Reviewer:}
\begin{itemize}
	\item Transactions on Information Forensics and Security, 2018
	\item Social Network Analysis and Mining (SNAM), 2017 (2 papers).
	\item Sustainability, 2017. 
\end{itemize}

{\bf Conference Reviewer:}
\begin{itemize}
\item International Joint Conference on Artificial Intelligence (IJCAI), 2018.
\item ACM SIGKDD Conferences on Knowledge Discovery and Data Mining (KDD), 2015, 2016.
\item AAAI Conference on Artificial Intelligence (AAAI), 2016.
\item International Conference on Autonomous Agents and Multiagent Systems, 2015. 
\end{itemize}


\section{\sc References} Available on request

\end{resume}
\end{document}




